\subsection{Test Suite Definition}
\label{sec:testSuiteDefinition}

Given an augmented GUI model $G^+$ the final phase of our technique generates a test suite with the following goal.

\noindent\textbf{Test objective:} \textit{A \textbf{compact} suite of tests to \textbf{comprehensively} test the interaction features of the given app under test.}

 A test is a sequence of actions (\ie a sequence of edges or a path in $G^+$) starting at the initial state $r$, with each action possibly followed by an oracle check. Therefore, a test can be represented as $\langle a_1, o_1, \dots, a_n, o_n \rangle$. Each $a_i \in A^+$ is any action (including those that exercise interaction features) allowed by the app's GUI. Each $o_i$ is either an oracle check or no operation. We assume that oracle checks are side effect free, \ie they do not change the state of the app.

We define a test adequacy criterion to concretize the notion of a ``comprehensive" test suite stated in the test objective. Since mobile apps are event-driven systems and interaction features are elements of the app's GUI, we develop a criterion that is motivated by the notions of path coverage and event-flow coverage used by previous work on GUI testing~\cite{memon2001coverage}. This is in contrast to code coverage-based criteria such as line or branch coverage, which would be more appropriate for functional testing of the software implementation rather than testing its high level platform features, as in our case. Intuitively, we say that a test suite covers a feature, if it contains tests to exercise and validate \textit{each possible instance} of exercising that feature on that app. Simply put, this implies exercising the feature in each GUI state. Given a test suite $T$, an interaction feature $f$ from a feature library $\mathcal{F}$ and an augmented GUI model $G^+ = \langle V, E^+, r, A^+, \mathcal{L}^+ \rangle$, as defined in Section~\ref{sec:modelAugmentation}, we define adequacy of $T$ in testing $f$ with respect to $G^+$ as follows.

\begin{mydef}[Interaction feature coverage]
\label{def:coverage}
A test suite $T$ covers a feature $f$ iff $\forall s \in S: \exists t \in T, t = \langle a_1, o_1, \dots, a_n, o_n  \rangle, \exists j, k, 0 \leq j < k \leq n$ such that $\langle a_1, \dots, a_j \rangle$ takes the app from the initial state $s_0$ to state $s$, $\langle a_{j+1}, \dots, a_k \rangle = \mathbf{u}_f$, and $o_k = O_f$.
\end{mydef}

Since there are no standard or widely accepted cost functions to optimize test suites we quantify the ``compactness"  of our test suite using the common sense observation that large test suites are hard to set up, execute, and maintain. The size of a test suite can be measured by the number of tests it contains as well as the cumulative number of operations (actions $a_i$) in the test suite as a whole.
We propose a customizable cost function that captures this.

%intuitively measures the time it takes to execute a test suite, including the set up and execution times. 
%Then we show how to leverage this cost function to traverse the model more effectively, in order to get smaller and faster to run test suites.

\begin{mydef}[Cost of a test suite]
\label{def:costfunction}
The cost of a test suite T is $cost(T) = \alpha * |T| + \beta * \Sigma_{t \in T} |t|$, where $\alpha$ and $\beta$ are positive co-efficients. \end{mydef}

Coefficient $\alpha$ measures the relative cost of developing and maintaining a suite, which scales with the number of tests in a suite. Coefficient $\beta$ quantifies the cost of executing actions and asserting oracles which is proportional to the number of operations.
