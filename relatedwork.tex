\Section{Related Work}
\label{relatedwork}

%% The oracle problem in software testing: specialized oracles and automatic oracles
Our work attempts to address the classic oracle
problem~\cite{Oracle:Howden78,Weyuker80}, in the context of mobile
apps. In practice, test oracles are typically specified manually,
often at the expense of substantial time and effort. There is a rich
body of work that aims to alleviate this long-standing problem by
automatically generating oracles. Software specification mining or
model inference techniques are often used for this purpose. A
comprehensive survey of API property inference techniques by Robillard
et al. describes many of these
techniques~\cite{robillard13tse}. Automated oracle generation
techniques usually generate general purpose oracles for functional
testing; they are not specific to any platform or class of software
applications or any aspects of software behavior. However, a recent
empirical study by Nguyen et al. on the cost and effectiveness of
automated oracles concludes that their false positive rate is often
prohibitively high for practical use~\cite{Nguyen:2013}.

%% Our work: Specialized oracles
Our proposed technique does not automatically generate general purpose
oracles but rather falls into a related body of work that uses
manually created oracles based on domain specific knowledge, that are
appropriately \textit{instantiated} during testing and used to test
very specific, sometimes non-functional, aspects of software
behavior. For example, the \textsc{Toddler} tool uses a hand-crafted
oracle that detects repetitive memory access patterns in loops to
identify performance bugs in software~\cite{toddler:ICSE2013}.  Our
previous work used differential testing~\cite{McKeeman1998} for oracle
automation in the context of testing web browser
implementations~\cite{ZaeemKhurshid2012} as well as detecting
%the \textsc{X-pert} tool detects 
cross-browser errors in web applications~\cite{xpert:ICSE2013}, \ie
discrepancies in web application behavior across different web
browsers, using test oracles specifically designed for these
domain-specific applications of differential testing.
%a test oracle specifically designed for this aspect of differential
%testing of web applications.
Our work in this paper
exploits characteristics of mobile apps and the mobile platform to
design oracles for testing an important class of user-interface
features of mobile apps.  Hu et al. also employ a specialized oracle
for testing Android mobile apps, which implements and checks the
Activity life-cycle
specification\footnote{\url{http://developer.android.com/training/basics/activity-lifecycle/index.html}}
for Android apps~\cite{Hu:2011:AST}.  However, this is one single
oracle, whereas our approach proposes an extensible framework that
spans a whole class of properties -- user-interaction features.

%% Automated test sequence generation for mobile apps
There is a growing body of research focused on automated testing of mobile applications. The proposed techniques span the complete gamut of technologies from random testing~\cite{Hu:2011:AST,Dynodroid:2013}, to symbolic-execution-based test-case generation~\cite{AnandFSE2012, MirzaeiJPF2012}, model-based testing, combinatorial testing, and combinations thereof~\cite{Nguyen:2012:ISSTA, collider2013}. However, the emphasis here is on generating test sequences to maximize code coverage, for the purpose of functional testing. The oracle problem is not directly addressed in these papers. It is implicit that the oracles would either be manually specified or use the simple oracle corresponding to catastrophic failure when the application crashes, hangs or otherwise throws an exception. By contrast, the focus of our approach is precisely to address the oracle problem, for a class of non-functional and platform-specific features of mobile apps. 

Our approach to test sequence generation falls under the broad area of model-based testing. The model may be manually specified or automatically extracted from the application under test. In fact there is a rich and active body of work on reverse-engineering such models from the user-interface of GUI applications~\cite{MemonWCRE2003}, web applications~\cite{Mesbah:2012:TWEB}, and more recently, mobile applications~\cite{AmalfitanoASE2012, Joorabchi:2012:WCRE, YangFASE2013}. However, our approach is independent of the method used to produce the model and is therefore orthogonal to these techniques.

The aim of model-based test sequence generation is to extract a suite
of concrete test cases based on the behavior represented in the
model. Most techniques in this category do this by heuristically
solving some variant of the NP-Hard \textit{minimum path cover
  problem}~\cite{PathCover:NtafosH79}, typically guided by some
supporting analysis and test suite sufficiency criteria. Memon et
al. propose several test adequacy criteria for GUI testing based on
coverage of events and event-sequences in the GUI
model~\cite{memon2001coverage}. Arlt et al. use a lightweight static
analysis to compute data dependencies between event-handlers of a GUI
application and use that to guide the choice of test sequences from
the GUI model~\cite{Arlt:2012:ISSRE}. Ganov et al., on the other hand,
focus on the problem of generating suitable values for the input
parameters of abstract test sequences extracted from a GUI model and
employ symbolic execution to compute these parameter
values~\cite{Ganov:2009:ICFEM}. Nguyen et al.  address the same
problem by using combinatorial testing techniques to embed
user-specified data values into abstract test
sequences~\cite{Nguyen:2012:ISSTA}. The aim of all the above
techniques is functional testing of the application and, more
specifically, to extract test cases which maximize coverage of the
application code. By contrast, our test sequence generation is
intended to exhaustively exercise a set of platform-specific
user-interaction features. This leads to different test targets,
cost-functions and ultimately a different set of model traversal
algorithms than those by pure functional testing approaches.
%\remark{Should we mention that we target errors of omission while code-coverage based approaches do not?}
%Our coverage criteria bears similarities with using states of a finite state machine~\cite{shehady1997method} or transitions between them~\cite{belli2001finite} as coverage criteria. Our work, however, is different in that we aim to cover transitions that are originally \emph{absent} from the GUI model, since, as discussed in Section~\ref{framework}, user-interaction features are usually excluded from such models to curb state space explosion.

 

%This paper draws from the following broad areas of research.
%\begin{itemize}
%	\item \textbf{Model Generation}: GUITAR~\cite{MemonWCRE2003}, Crawljax~\cite{Mesbah:2012:TWEB}, AndroidRipper~\cite{AmalfitanoASE2012}, iOSCrawler~\cite{Joorabchi:2012:WCRE}, ORBIT~\cite{YangFASE2013}.
%	\item \textbf{Model-based Test Generation}: Selection of test sequences based on a model: \cite{Arlt:2012:ISSRE}, \cite{Nguyen:2012:ISSTA}, \cite{Ganov:2009:ICFEM}, \cite{Takala:2011}.
%	\item \textbf{Testing of Mobile Software}: Using random testing~\cite{Hu:2011:AST}, symbolic execution~\cite{AnandFSE2012}, \cite{MirzaeiJPF2012}.
%	\item \textbf{Oracles for Software Testing}: First introduced in \cite{Oracle:Howden78} and developed in \cite{Weyuker80}. Different methods of constructing oracles: from program documentation~\cite{Peters:ISSTA94}, from method post-conditions~\cite{postCondOracle:Meyer09} or using statistical methods~\cite{StatOracle:Mayer04}.  
%\end{itemize}







